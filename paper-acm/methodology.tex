\section{Methodology}
\label{sec:methodology}

To be able to determine how fingerprintable a browser is, we need to
determine the features it supports when a web page is visited by the
user. Similarly, we need to understand what features are supported by
a specific version because attackers typically target such features in
attacks (e.g., a bug in the video access functionality might be
exploited). Hence, to answer the research questions we pose in this
study, we need to be able to figure out exactly what features are
supported by each browser version under analysis, and we also need
access to vulnerability data for each browser. In this section, we
describe the methodology we followed in this work, and explain how we
created the datasets we used in our analyses.

\subsection{Feature Gathering}
\label{sec:feature-gathering}

In order to collect ``feature'' sets from Firefox and Chrome, we
redirect the browser under analysis to a JavaScript-instrumented web
page. We use the term \textit{feature} to describe JavaScript objects,
methods, and property values built into the global namespace of the
browser's JavaScript implementation (i.e., the \texttt{window}
object). Clearly, this definition is JavaScript-centric. However, it
is unambiguous and naturally scalable. That is, we can automate the
collection of features from many different browser implementations
using standard scripting and crawling techniques. When the
instrumented page is loaded by the browser, our JavaScript is executed
that then probes and iterates through the features supported by the
browser. This is done by using JavaScript to traverse the tree of
non-cyclic JavaScript object references accessible from a pristine
(i.e., unmodified) \texttt{window} object, and collecting the full
feature names encountered during the traversal. Each feature name
comprises the sequence of property names leading from the global
object to a given built-in JavaScript value. The traversal code is
careful to not modify this object (which doubles as the global
variable namespace) in any way, to avoid contaminating the resulting
set of feature names. Captured feature sets are then stored in a
database, tagged with identifying metadata such as the browser's
User-Agent string.

We use the terms \textit{browser features}, as defined in this section, and \textit{JavaScript APIs} interchangeably in our work.


%%%%%%%%%%%%%%%%%%%%%%%%%%%%%%%%%%%%%%%%%%%%%%%%
%%% Browser Fingerprinting APIs %%%%%%%%%%%%%%%%
%%%%%%%%%%%%%%%%%%%%%%%%%%%%%%%%%%%%%%%%%%%%%%%%
\subsection{Browser Fingerprinting APIs}
\label{sec:fp-apis}

We conduct an in-depth analysis in order to determine which browser features are associated with fingerprinting. We use this list of suspicious APIs in our measurements in Section~\ref{sec:analysis} to quantify \textit{fingerprintability}: the ratio of browser features in a browser version that are associated with fingerprinting techniques. We describe in the following how we conducted the list of suspicious browser features that are related to browser fingerprinting.

Our list of suspicious browser fingerprinting API contains a total of 313 JavaScript APIs. These APIs are considered suspicious because the purpose of using these API depends on the programmer who writes the source code. 
We call this list \textit{suspicious fingerprinting APIs} in this paper.
In Panopticlick's research~\cite{panopticlick}, browser fingerprinting is achieved through a combination of APIs that seem innocent, such as \texttt{Navigator.plugins}, \texttt{Navigator.userAgent}, and \texttt{Screen.colorDepth}. These APIs are designed with their original objectives. However, they are chosen to fingerprint browsers due to their alternative functionality in collecting information to narrow down the scope of visited users. Based on our approach taken to collect APIs, there is no way to determine whether the source code is doing browser fingerprinting without the acknowledgment of the writer of the code. 

We use two methods to assemble the list of fingerprinting APIs: literature review and experimental analysis. Literature review, the foundation of the API list, is composed of four core fingerprinting papers, Panopticlick~\cite{panopticlick}, AmIUnique~\cite{amiunique}, Hiding in the Crowd~\cite{hidinginthecrowd}, and FPDetective~\cite{fpdetective}. This analysis resulted in approximately 10\% of the list of suspicious fingerprinting APIs. Some of the APIs are directly mentioned in these papers and the others are modified to match standard APIs\footnote{\url{https://developer.mozilla.org/en-US/docs/Web/API}} with the same functionality. The concepts of Canvas, WebGL, Font fingerprinting are introduced among these APIs. These concepts lead to the next turn of investigation of papers which are Cookieless Monster~\cite{cookiemonster-SP13} and Pixel Perfect~\cite{mowery2012pixel}. This investigation does not bring more APIs but a direction to experimental analysis. 

The experimental analysis consists of two stages, collecting APIs by crawling websites and extracting suspicious APIs from the crawling data. In terms of data collecting, the workflow is the same as the one in VisibleV8~\cite{vv8-imc19}. A customized crawler was driven to visit all websites in the Easylist~\cite{Easylist} domain file that contains 13,241 domains. Then, the raw logs generated by VV8 were gathered and the VV8 post processor was applied to process the raw data. After removing duplicate and non-standard APIs, the APIs usage of 8,682 domains with 56,828 origins was collected. Non-standard APIs indicate ones that are not listed in the WebIDL~\cite{webidl} data package. In other words, VV8 and its post processor were adopted to aggregate and summarize standard JS API usage of the target domains.

While collecting APIs from the wild, the API suspicious list was extended through crawling on \url{panopticlick.eff.org}, \url{amiunique.org}, and \url{browserleaks.com} websites. These websites are explicitly marked as browser fingerprinting websites. Therefore, augmenting suspicious fingerprinting APIs among these websites is more efficient than a random walk on the enormous JS API pool.

The next step is to implement a manual analysis to check every API utilized by these three websites. First, we search for information and usage of an API on https://developer.mozilla.org/en-US/docs/Web/API. Then, determine whether an API fingerprints users based on the information the API conveys. That is to say, an API is classified as a suspicious fingerprinting API if it can provide the information to filter certain users out. For example, there are two users with distinct user agents. By calling Navigator.userAgent, the programmer should be able to distinguish between these two users. Navigator.userAgent can be recognized as a fingerprinting API in this case. The majority of suspicious fingerprinting APIs comes from the manual analysis and the idea of categorizing fingerprinting APIs is incited by browserleaks.com website. 

The last step is to manually search for more fingerprinting APIs with the keyword. Namely, in Canvas fingerprinting, most APIs include the ``Canvas'' or ``CanvasRendering''. A program was created to filtrate APIs that contain ``Canvas'' or ``CanvasRendering'' among APIs of 8k crawled domains. The same pattern also applies to BatteryManager, WebGLRenderingContext, and SpeechSynthesis. Meanwhile, the fingerprint2.js was reviewed to supplement the suspicious fingerprinting API list.  

There are limitations to the methods we used for constructing a suspicious fingerprinting API list. First and foremost, this list only provides a partial view of full fingerprinting APIs. To the best of our knowledge, there is no complete table of fingerprinting APIs and there could be research in this direction. The second limitation is during the manual analysis. There could be misconceptions between the API usage provided by Mozilla web APIs page and the way programmers exploit them. Lastly, part of JS APIs is filtered out by the VV8 post processor. This can be improved by using a larger set of WebIDL data or precisely use the aggregated raw APIs. 

We plan to make our list of fingerprinting APIs publicly available upon publication.

\subsection{Browser Testing Platform}

In this work, we decided to target Google Chrome and Mozilla
Firefox browsers as they are well-known, popular browsers that have
millions of users. Also, these browsers possess distinct codebases
(i.e., unlike Microsoft Edge that is based on Google Chrome). We
gathered a copy of every major Firefox and Chrome version that was
released during the March 2016 to April 2020 timeframe (i.e., Chrome
versions 49 to 81, and Firefox versions 45 to 75).

To individually connect each browser version to our instrumented
feature gathering web application, we mainly used the BrowserStack web
service~\cite{browserstack}. BrowserStack is a cloud-based web and mobile testing platform
that enables developers to test their websites and mobile applications
across on a wide range of browsers, operating systems, and real mobile
devices. If a specific browser version or configuration was not
available on BrowserStack, we developed and used automation scripts to
instrument and run the browser instances on a desktop computer running
Windows 10.

\subsection{Vulnerability Information Gathering}

One major source of information for security vulnerabilities is the
CVE (Common Vulnerabilities and Exposures) dataset that is hosted by
MITRE. A CVE entry contains several fields with standardized
text/definitions that represent a publicly disclosed cybersecurity
vulnerabilities and exposures. Each CVE entry has a unique CVE
identifier, a general description, and several references to one or
more external information sources of vulnerability.

For our study, we used the CVE data from the National Vulnerability
Database (NVD) that is provided by the National Institute of Standards
and Technology (NIST). For each CVE entry in this dataset, we
extracted the description, the affected product its specific version
number, and the severity of the vulnerability. Then, we parsed this
data and generated a CVE entry list for each browser version in our
dataset. This new dataset was the basis of the vulnerability and
feature analysis in this paper.
