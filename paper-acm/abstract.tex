\begin{abstract}

  Web browsers have become important tools in our daily lives.
  Millions of users use web browsers for different purposes such as
  social media, online shopping, or surfing the web. Some of these
  services use browser fingerprinting to track and profile their users
  which can be in contrast with their web privacy. At the same time,
  browsers are also being targeted by attackers because they are
  attractive platforms to compromise.

  In this paper, we perform an empirical analysis of a large number of
  browser features intending to evaluate fingerprinting possibility,
  and also vulnerability issues that are based on different browser
  features. By analyzing 33 Google Chrome and 31 Mozilla Firefox major
  browser versions released through 2016 to 2020, we discover that all
  of these browsers have unique feature sets which makes them
  different from each other. By comparing these features to the
  fingerprinting APIs presented in the literature that have appeared in
  this field, we conclude that all of these browser versions are
  uniquely fingerprintable. Our results show an alarming trend that
  browsers are becoming more fingerprintable over time because newer
  versions have more fingerprintable APIs in them. Another key
  discovery of our analysis is that unlike the popular belief about
  software bloating causing an increased attack surface, the
  number of vulnerabilities in browsers are not directly related to
  the number of new features being introduced by those browsers. In
  fact, our measurements suggest that browsers are becoming more secure
  over time although they are supporting more features than in the past.

\end{abstract}
