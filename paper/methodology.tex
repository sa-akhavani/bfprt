\section{Methodology}
\label{sec:methodology}

To be able to determine how fingerprintable a browser is, we need to
determine the features it supports when a web page is visited by the
user. Similarly, we need to understand what features are supported by
a specific version because attackers typically target such features in
attacks (e.g., a bug in the video access functionality might be
exploited). Hence, to answer the research questions we pose in this
study, we need to be able to figure out exactly what features are
supported by each browser version under analysis, and we also need
access to vulnerability data for each browser. In this section, we
describe the methodology we followed in this work, and explain how we
created the datasets we used in our analyses.

\subsection{Feature Gathering}

In order to collect ``feature'' sets from Firefox and Chrome, we
redirect the browser under analysis to a JavaScript-instrumented web
page. We use the term \textit{feature} to describe JavaScript objects,
methods, and property values built into the global namespace of the
browser's JavaScript implementation (i.e., the \texttt{window}
object). Clearly, this definition is JavaScript-centric. However, it
is unambiguous and naturally scalable. That is, we can automate the
collection of features from many different browser implementations
using standard scripting and crawling techniques. When the
instrumented page is loaded by the browser, our JavaScript is executed
that then probes and iterates through the features supported by the
browser. This is done by using JavaScript to traverse the tree of
non-cyclic JavaScript object references accessible from a pristine
(i.e., unmodified) \texttt{window} object, and collecting the full
feature names encountered during the traversal. Each feature name
comprises the sequence of property names leading from the global
object to a given built-in JavaScript value. The traversal code is
careful to not modify this object (which doubles as the global
variable namespace) in any way, to avoid contaminating the resulting
set of feature names. Captured feature sets are then stored in a
database, tagged with identifying metadata such as the browser's
User-Agent string.

\subsection{Browser Testing Platform}

In this work, we decided to target the Google Chrome and Mozilla
Firefox browsers as they are well-known, popular browsers that have
millions of users. Also, these browsers possess distinct code bases
(i.e., unlike Microsoft Edge that is based on Google Chrome). We
gathered a copy of every major Firefox and Chrome version that was
released during the March 2016 to April 2020 timeframe (i.e., Chrome
versions 49 to 81, and Firefox versions 45 to 75).

To individually connect each browser version to our instrumented
feature gathering web application, we mainly used the BrowserStack web
service. BrowserStack is a cloud-based web and mobile testing platform
that enables developers to test their websites and mobile applications
across on a wide range of browsers, operating systems, and real mobile
devices. If a specific browser version or configuration was not
available on BrowserStack, we developed and used automation scripts to
instrument and run the browser instances on a desktop computer running
Windows 10.

\subsection{Vulnerability Information Gathering}

One major source of information for security vulnerabilities is the
CVE (Common Vulnerabilities and Exposures) dataset that is hosted by
MITRE. CVE is a dictionary of publicly disclosed cybersecurity
vulnerabilities and exposures. Each CVE entry has a unique CVE
identifier, a general description, and several references to one or
more external information sources of the vulnerability.

For our study, we used the CVE data from the National Vulnerability
Database (NVD) that is provided by the National Institute of Standards
and Technology (NIST). For each CVE entry in this dataset, we
extracted the description, the affected product its specific version
number, and the severity of the vulnerability. Then, we parsed this
data and generated a CVE entry list for each browser version in our
dataset. This new dataset was the basis of the vulnerability and
feature analysis in this paper.

