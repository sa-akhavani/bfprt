\section{Methodology}
\label{sec:methodology}

In order to be able to answer how browser features are evolving over time, and also how these features are being attacked, we need to have access to a huge amount of browser feature data and also browser vulnerability data. In the next sections, we explain the process
we applied to collect and classify vulnerability reports and exploit descriptions.

\subsection{Feature Gathering}
\subsubsection{Feature Extractor Script}
\subsubsection{Browser Testing}
\ali{1- Browserstack 2-Automated Platform}


\subsection{Vulnerability Gathering}
One major source of information for security vulnerabilities is the CVE dataset,
which is hosted by MITRE. According to MITRE’s FAQ, CVE is not
a vulnerability database but a vulnerability identification system that ‘aims to
provide common names for publicly known problems’ such that it allows ‘vulnerability databases and other capabilities to be linked together’. Each CVE entry
has a unique CVE identifier, a status (‘entry’ or ‘candidate’), a general description, and a number of references to one or more external information sources of
the vulnerability. These references include a source identifier and a well-defined
identifier for searching on the source’s website. Vulnerability information is provided to MITRE in the form of vulnerability submissions. MITRE assigns a CVE
identifier and a candidate status. After the CVE Editorial Board has reviewed
the candidate entry, the entry may be assigned the ‘Accept’ status.
For our study, we used the CVE data from the National Vulnerability Database
(NVD) which is provided by the National Institute of Standards and Technology (NIST).