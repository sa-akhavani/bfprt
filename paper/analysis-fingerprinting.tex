We wanted to answer two different things about browser fingerprinting in our paper. The questions are as follows. Can we say that all browsers are fingerprintable? Are browsers becoming more fingerprintable?

To answer the first question, we created a feature set for each browser version that we had. A feature set is a set of browser features that exist in a specific browser version. After creating the feature set for each browser version in our study, we compared these sets to each other and realized that all of these sets are unique. This means that there are no two browsers that have the same feature set. This means that all of these browsers are fingerprintable. The reason for this finding is that browser vendors keep removing and adding features to their newer versions so that the feature set for each version becomes different than the others. So the answer to our question is yes. All of the browser versions in our study were fingerprintable.


In order to answer the second question, we tried to compare the feature sets to each other. We see that vendors especially Google Chrome are adding lots of features to their newer versions but they do not remove features as much as they add. We see that the feature set for each browser is converging and we are tending to the homogeneity of browser features. The unique feature set is getting smaller in almost every new browser version. The unique feature set is the set of features that make a specific browser version unique; either by being in the feature set or by not being in the feature set. \ali{maybe explain this more}
So based on our findings, we cannot say that browsers are becoming more fingerprintable.They are becoming less fingerprintable.

\ali{Mention the size of the smallest and biggest feature set}

\ali{Generate a chart to show this finding}