\section{Related Work}
\label{sec:related-work}

Our work focuses on the intersection of browser evolution and browser
fingerprinting.

\textbf{Browser evolution} The first web browser,
WorldWideWeb~\cite{WorldWideWeb}, was developed in 1990 by Tim
Berners-Lee. That browser did not have JavaScript, did not support
cookies and users could not adapt their browser with extensions. All
these features and thousands more were introduced in browsers over
time, matching the needs of the ever-evolving web.

Synder et al.~\cite{snyder-imc16} use a similar method to us to
collect browser features by using the web API and extracting different
kinds of JavaScript functions. They measure browser feature usage
among Alexa's popular websites and also how many security
vulnerabilities have been associated with related browser
features. However, they do not aim to measure fingerprintability of
different browsers which is one of the main goals of our paper. In
another work by Snyder~\cite{snyder2017most}, a cost-benefit approach
to improving browser security was conducted.  Our work focuses on how
browsers have become more fingerprintable over time based on the
features they introduce, taking a new perspective on the privacy and
security costs that the browser evolution brings.

Recent work has focused on methods to automatically reduce the
functionality of the browser at the binary level. Chenxiong et
al.~\cite{slimium-ccs2020} proposes a debloating framework for the
browser that removes unused features. Our work is complementary to
debloating efforts of the browser, as we focus on which browser
features affect the users' privacy the most. Also, our work shows that
the debloating of browsers might not really be necessary as there does
not seem to exist a correlation between the number of features added
to the browsers over time, and how insecure they become.

\textbf{Browser fingerprinting} There have been a number of studies on
browser fingerprinting and browser bloating. The first large-scale
study on browser fingerprinting was conducted by
Eckersley~\cite{Eckersley}. Eckersley showed that a wide range of
properties in a user's browser and the installed plugins can be
combined to form a unique fingerprint. His study made us eager to see
what is happening in the world of browser features, and to try to
analyze the impact of different browser features on creating unique
user fingerprints.

Browser fingerprinting can be done by using different methods. Cao et
al.~\cite{Cao} created user fingerprints by using OS-level features
from screen resolution to the number of CPU cores. They also measure
the uniqueness of different browser types by analyzing its OS-level
features.

Olejnik et al.~\cite{Olejnik} show that one way of fingerprinting a
browser is using web history. In this method, there is no need for a
client-side state. However, note that this method is no longer
possible because browser vendors have fixed this issue and (i.e.,
extracting user history is not possible as before).

Nikiforakis et al.~\cite{cookiemonster-SP13} showed how tracking has
moved from using cookies (stateful) to browser fingerprinting
(stateless) on the web. Mowery et al.~\cite{mowery2012pixel}
demonstrated how the \texttt{canvas} HTML5 feature can be abused for
browser fingerprinting based on the differences in rendering images on
different GPUs. Starov et al.~\cite{extbloat-www2019} measured how
bloated browser extensions are in terms of the artifacts that they
inject in visited pages, and can be used to identify the presence of
the users' installed extensions.  Trickel et al.~\cite{cloakx-sec19}
proposed a defense mechanism against identifying installed browser
extensions in users' browsers based on artifacts that reveal their
presence on the visited pages.

In light of the prior research on browser fingerprinting, our aim was
to collect data and analyze the trends, and to see whether we are
becoming better at managing browser fingerprinting (or if this privacy
issue is becoming worse as new features are being introduced in new
browser versions).
