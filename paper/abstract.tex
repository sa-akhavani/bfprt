\begin{abstract}
Web browsers have become important services in our daily lives. Millions of users use web browsers for different purposes such as social media, online shopping, or surfing the web. Some of these services use browser fingerprinting to track and profile their users which can be in contrast with their web privacy. 

In this paper, we perform an empirical analysis of a large number of browser features intending to evaluate fingerprinting possibility and also vulnerability issues that are based on different browser features. By analyzing 33 Google Chrome and 31 Mozilla Firefox major browser versions released through 2016 to 2020 we discover that all of these browsers have unique feature sets which makes them different from each other. By comparing these features to the fingerprinting APIs presented in different papers in this field, we conclude that all of these browser versions are uniquely fingerprintable. Our results show a worrisome trend that browsers are becoming more fingerprintable over time because newer versions have more fingerprintable APIs in them. The other discovery of our analysis is that the number of browser vulnerabilities is not related to the number of features in that version and although newer browsers are having more features, they are becoming more secure and have less reported CVEs.
\end{abstract}
