\section{Research Questions}
\label{sec:introduction}

In this paper, by performing an automated analysis, we attempt to
answer the following research questions:

\begin{enumerate}
  
\item {\em Are all major, popular Firefox and Chrome browsers
    fingerprintable?} Our results show that the feature set for each
  Firefox and Chrome browser version is unique. There exist multiple
  APIs in every browser version that we have analyzed that can be used
  for fingerprinting. By extracting all the features supported by a
  browser using API calls, we can indeed uniquely identify each
  browser version.

\item {\em Are Firefox and Chrome browsers becoming more
    fingerprintable over time?} One of the major conclusions of our
  study is that the number of APIs one can use in the newer versions
  of Chrome and Firefox is larger than in older versions. Hence, newer
  browser versions are even more fingerprintable than previous
  versions, and our findings suggest that this trend is likely to
  continue. As a result, privacy might be an even more significant
  concern in the future for browser users.
    
\item {\em What ``lifespan profiles'' can we cluster browser features
    into? Are there any``permanently removed'' features? If so, how
    did their life cycle look like?} Our results show that we can
  categorize browser features based on their lifespan into three main
  categories. We observe that most of the features are added
  permanently, and are not removed over time -- indicating that
  browsers are indeed becoming more ``bloated'' as they evolve.

\item {\em With respect to browser bloating, how does Firefox compare
    to Chrome?} In our study, we were able to map the number of unique
  features for major versions of Firefox and Chrome. The results
  suggest that Chrome is introducing more features over time than
  Firefox, but that both browser vendors have shown a significant
  increase in the total number of features they support per version
  since 2016. Compared to Firefox, Chrome tends to introduce more new
  features and keep them longer than Firefox.
 
\item {\em Is there a correlation between the number of features
    available in a browser (i.e., how ``bloated'' the browser is) and
    the number of vulnerabilities that exist on that browser?} Our
  data suggests that there is no direct correlation between the number
  of features that a browser version supports and the number of
  vulnerabilities in that version. Although browsers are indeed
  becoming more ``bloated'' over time, at the same time, their code
  base seems to be becoming more secure.    
    
\end{enumerate}

