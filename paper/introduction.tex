\section{Introduction}
\label{sec:introduction}

A browser fingerprint is a set of information related to a user’s device such as hardware information, operating system, timezone, browser and its configuration, and much more. Browser fingerprinting refers to the
process of collecting information through a web browser to build a fingerprint of a device.

Another terminology that we use in this paper is bloating. We say that a browser is bloated when it has lots of unused or unnecessary features in it.

In this paper, by performing multiple analysis on the data that we have collected, we attempt to answer
the following questions:

\begin{enumerate}
  \item Can we say that all browsers are fingerprintable?
  Yes. We show that the feature set for each browser version in Chrome and Firefox is unique. Also, there exist multiple fingerprinting APIs in every browser version we have analyzed. So we can identify each browser version by extracting its features and all browsers are fingerprintable.
%   \item What "lifespan profiles" can we cluster browser features into? Are there any "permanently removed" features, and if so, how old/established were they? How long is the feature lifespan of removed features?
  \item With respect to bloating, if we map the number of features per browser, how does the trend look? The number of features per browser is increasing. It is more intense in Chrome and they are adding much more features than Firefox but both of the browser vendors have had a significant increase in the number of features per browser in the 4 year period that we analyzed.
  \item Who is leading and who is following in feature introduction among browser vendors? Are we seeing some major differences in terms of feature introduction and removal between Firefox and Chrome? The feature introduction and removal are not similar between Chrome and Firefox. Firefox tends to keep its feature numbers steady. They remove a bigger portion of their features compared to Chrome. But chrome tends to add more features and keeps them in the browser. So there is not a similarity between Chrome and Firefox in terms of feature introduction.
  \item Is there a correlation between the number of features in a browser and the number of vulnerabilities that exist on that browser? Based on our data and analysis, there is not a valid correlation between the number of browser features and the number of vulnerabilities. Although we are having many more features in the newer browsers, they are becoming more secure and the number of features does not affect the security of browsers.
  \item Are browsers becoming more fingerprintable? Yes. One major finding of our analysis is that number of fingerprinting APIs in newer browsers is bigger than the older versions so we can say that newer browsers are becoming more fingerprintable. versions.
\end{enumerate}

The rest of the paper is organized as follows: The next section describes our methodology and data gathering technique. Section 3 presents a different analysis based on the feature reports and their relation to browser vulnerability. In Section 4, we present the related work and then briefly conclude the paper in Section 5.

